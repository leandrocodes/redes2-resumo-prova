\documentclass[a4paper, 12pt]{article}

\usepackage[portuges]{babel}
\usepackage[utf8]{inputenc}
\usepackage{amsmath}
\usepackage{indentfirst}
\usepackage{graphicx}
\usepackage{multicol,lipsum}

\begin{document}
%\maketitle

\begin{titlepage}
	\begin{center}
	
	%\begin{figure}[!ht]
	%\centering
	%\includegraphics[width=2cm]{c:/ufba.jpg}
	%\end{figure}

		\large{Universidade Federal de Mato Grosso do Sul}\\
		\large{Campus de Ponta Porã}\\ 
		\vspace{15pt}
        \vspace{95pt}
        \textbf{\LARGE{Relatório - Camada de Rede }}\\
		%\title{{\large{Título}}}
		\vspace{3,5cm}
	\end{center}
	
	\begin{flushleft}
		\begin{tabbing}
			Aluno: Leandro Viana Martins\\
			Professor: Dr. Dionisio Machado Leite\\
	\end{tabbing}
 \end{flushleft}
	\vspace{1cm}
	
	\begin{center}
		\vspace{\fill}
			Setembro\\
            2019
	\end{center}
\end{titlepage}
%%%%%%%%%%%%%%%%%%%%%%%%%%%%%%%%%%%%%%%%%%%%%%%%%%%%%%%%%%%

\newpage
\section{Introdução}
Na camada de rede, são implementados serviços de comunicação hospedeiro a hospedeiro, diferentemente da camada de transporte. 
Portanto, há "pedaços" da camada em cada um de seus hospedeiros e roteadores na rede.
\\
\\
É necessário distinguir as funções de \textbf{repasse e roteamento}, onde o primeiro envolve a transferência de um pacote de um enlace
de entrada para um elance de saída. Já o segundo, envolve todos os roteadores de uma rede, onde os caminhos que os pacotes 
percorrem serão determinados por protocolos.

\subsection{Repasse e Roteamento}
O papel da camada de rede é \textbf{ transportar os pacotes de um hospedeiro origem até um hospedeiro destinatário}.

    \subsubsection{Repasse}
        Quando um pacote chega num roteador, este deve encaminhar até a saída apropriada.
        Dado um pacote — proveniente de um hospedeiro H1 — que chega ao roteador R1 com destino a um hospedeiro H2, deve ser 
        repassado ao roteador seguinte até chegar em seu destino.

    \subsubsection{Roteamento}
        A camada de rede deve determinar a rota em que um pacote flui de seu remetente à seu destinatário. Os responsáveis por essas
        atividads são os \textbf{algoritmos de roteamento}. Nesse caso, um algoritmo definiria o caminho entre dois hospedeiros em que
        os pacotes seguirão.

\subsection{Modelos de Serviço de rede}

Um modelo de serviço de rede define características pertinentes ao transporte dos dados \textbf{ fim a fim entre bordas da rede}.
Visto isso, o modelo poderia prover alguns serviços como: \textit{Entrega garantida, entrega garantida com atraso limitado, 
entrega de pacotes em ordem, largura de banda mínima garantida, Jitter e serviços de segurança.}

    \subsubsection{Arquitetura de rede: Internet}
    Possui o modelo de serviço \textit{melhor esforço}, com \textbf{nenhuma} garantia de \textit{largura de banda} e \textit{contra perda}.
    Seu \textit{ordenamento} é feito em \textbf{qualquer ordem possível} sem manter \textit{temporização} com \textbf{nenhuma} 
    \textit{indicação de congestionamento}

    \subsubsection{Arquitetura de rede: ATM}
    Possui o modelo de serviço \textit{CBR}, com \textbf{taxa constante} de garantia de \textit{largura de banda} e possuí 
    \textit{garantia contra perda}.
    Seu \textit{ordenamento} é feito em \textbf{ordem} mantendo \textit{temporização} com \textbf{garantia de que não ocorrerá} 
    \textit{congestionamento}

    \subsubsection{Arquitetura de rede: ATM}
    Possui o modelo de serviço \textit{ABR}, com \textbf{taxa mínima} de garantia de \textit{largura de banda} e \textbf{não possuí} 
    \textit{garantia contra perda}.
    Seu \textit{ordenamento} é feito em \textbf{ordem}, mas sem \textit{temporização} com \textbf{garantia de congestionamento} 

\newpage

\section{Redes de Circuitos Virtuais e Datagramas}
A camada de rede também oferece dois serviços para as aplicações, semelhantes aos que a camada de transporte oferece. Que são: serviços 
de camada de rede orientados para conexão e não orientados para conexão.
\\
\\
Um serviço orientado para conexão começa com uma apresentação entre os hospedeiros de origem e destino. Já o não orientado, não possui 
apresentação alguma.
\subsection{Redes de Circuitos Virtuais}
Um circuito virtual é um serviço orientado a conexão, onde trata-se basicamente de um caminho (série de enlaces e roteadores) entre 
hospedeiros de origem e destino. O qual contém um número para cada enlace e registros da tabela de repasse em cada roteador ao longo do caminho.

    \subsubsection{Estabelecimento do Circuito virtual}
    O estabelecimento de um circuito virtual é consistido por três fases:
        \\\\*
        I. \textbf{Estabelecimento de CV}: A camada de transporte remetente contata a camada de rede especificando o endereço do destinatário.
        Depois, a camada de rede determina o caminho entre o remetente e destinatário.
        \\\\*
        II. \textbf{Transferência de dados}: Fluxo dos pacotes entre os hospedeiros.
        \\\\*
        III. \textbf{Encerramento do CV}: O encerramento começa quando um dos hospedeiros avisa à camada de rede seu desejo de desativar o CV.
        A camada de rede então, informa ao sistema final do outro lado sob o fim da conexão e atualiza as tabelas de repasse de cada um dos
        roteadores.

\subsection{Rede de Datagramas}
Uma rede de Datagramas é um serviço não orientado a conexão, quando um sistema final quer enviar um pacote, ele informa no pacote
o endereço do de destino e posteriormente o envia pra dentro da rede. Nesse, cenário, a rota em que fluxo dos pacotes seguirão está definida
na tabela de repasse dos roteadores. Nesta rede, os pacotes atraavessam vários roteadores, onde cada roteador utiliza o endereço de destino do 
pacote para transmitir a uma saída.

\newpage
\section{O que há dentro do roteador?}
O roteador possui várias portas de entradas e saídas, que são regidas pelo elemento de comutação, o qual é constituido por um processador de
roteamento.

    \subsection{Porta de Entrada}
    O processamento na porta de entrada começa pela terminação de linha, em seguida é efetuado o desencapsulamento dos pacotes, logo após a consulta e o
    enfileiramento de repasse para a porta apropriada para poder entregar ao elemento de comutação.

        \subsection{Elemento de Comutação}
        Elemento/dispositivo que efetua a permutação da porta de entrada para a porta de saída.
        \\\\*
        Essa comutação pode ser feita de várias formas: \textit{por memório, barramento ou por uma rede de interconexão}

        \subsubsection{Comutação por Memória}
        O primeiros roteadores, em sua maioria, eram computadores tradicionais. Portanto, a comutação entre as portas de entrada e saída era realizada 
        sob o controle direto da CPU. Nesse cenário, as portas funcionavam como dispositivos de entrada e saída de um SO tradicional.

        \subsubsection{Comutação por Barramento}
        Nessa abordagem, as portas de entrada transferem um pacote diretamente para a porta de saída por um barramento compartilhado sem a intervenção
        do processador de roteamento. No entanto, somente um pacote por vez pode ser transferido, uma vez que o barramento encontra-se ocupado.

        \subsubsection{Comutação por Rede de Interconexão}
        Quase como a comutação por barramento, o que diferencia-se é a quantidade de barramentos, nesse caso são utilizado 2 ou mais, onde cada um 
        desses barramentos conectam \textbf{n} portas de entrada com \textbf{n} portas de saída

\subsection{Porta de Saída}
O processamento na porta de saída começa com o gerenciamento de fila feito por um buffer, em seguida é efetuado o encapsulamento dos pacotes e por 
fim há a terminação de linha.

    \subsubsection{Onde ocorre a formação de fila?}
    As filas podem se formar tanto nas portas de entrada, quanto nas de saída. A formação 
    da mesma, depende de alguns fatores: carga de tráfego, da velocidade relativa do 
    elemento de comutação e da taxa linha.
    \\
    \\
    Um problema bastante conhecido é o \textbf{Bloqueio de cabeça da fila}, onde um datagrama enfileirado na frente da fila, impede
    que outros datagramas sigam fila adiante.

\newpage

\section{Protocolo de Internet (IP)}
Existem alguns componentes que formam a camada de rede na Internet. O primeiro é o IP e o
segundo é o roteamento, que determina o caminho que um datagrama segue desde a origem ao
seu destino. Os protocolos de roteamento calculam as tabelas de repasse que são usadas
para transmitir pacotes pela rede. O último componente é um dispositivo de comunicação
de erros em datagramas e para atender requisições de certas informações da camada de 
rede (ICMP). 

    \subsection{Formato do datagrama}
    O datagrama desempenha um papel central na Internet.

        \subsubsection{IPV4}
        \textbf{\textit{Número da versão}}: São quatro bits que especificam a versão do protocolo IP do datagrama.
        \\\\
        \textbf{\textit{Comprimento do cabeçalho}}: O datagrama IPV4 pode conter um número variável de opções, esses quatro bits
        são necessários para determinar onde os dados começam no pacote.
        \\\\
        \textbf{\textit{Tipo de serviço}}: Os bits de tipo de serviço servem para diferenciar os diferentes tipos de datagrama IP
        que devem ser distinguidos uns dos outros.
        \\\\
        \textbf{\textit{Comprimento do datagrama (Fragmentação)}}:
        É o comprimento total do datagrama IP medido em bytes. Esse campo tem 16 bits, então o tamanho máximo teórico do datagrama
        IP é 65.535 bytes. No entano, raramente são maiores que 1.500 bytes.
        \\\\
        \textbf{\textit{Tempo de vida}}: É incluído para garantir que datagramas não fiquem circulando para sempre. Seu TTL é 
        decrementado em uma unidade por cada processamento feito por roteadores, quando esse valor chega a zero, ele é descartado.
        \\\\
        \textbf{\textit{Protocolo}}: É usado somente quando um datagrama IP chega a seu destino final. O valor desse campo indica
        o protocolo de cama de transporte específico ao qual a porção de dados desse datagrama IP deverá ser passada.
        \\\\
        \textbf{\textit{Soma de verificação do Cabecalho}}: Utilizado para auxiliar o roteador na detecção de erros de bits em
        um datagrama IP recebido.
        \\\\
        \textbf{\textit{Endereços IP fonte e destino}}: Quando uma fonte cria um datagrama, insere seu endereço IP e o do destinatário.
        Muitas vezes o hospedeiro fonte determina o endereço do destinatário por meio de DNS.
        \\\\
        \textbf{\textit{Opções}}: Campo em que permite que um cabeçalho IP seja ampliado.
        \\\\
        \textbf{\textit{Dados}}: Contém o segmento de camada de transporte a ser entregue ao destino (TCP ou UDP). Mas, também pode
        carregar outros tipos de dados, como por exemplo o ICMP.

    \subsection{Fragmentação do Datagrama}
    A fragmentação de um datagrama consiste na "separação" e na reconstrução de datagramas IP, de forma em que atenda a quantidade
    máxima de bits que pode ser trafegada com um único pacote em um determinado enlace. Essa tarefa é desempenhada pelo roteador.
    \\\\
    O IPV6 não implementa a fragmentação, permitindo que as aplicações e o protocolo TCP desempenhem tal qual função ou algo equivalente.
    \\\\
    A fragmentação não é muito performática, pois o roteador deve ficar fragmentando e desfragmentando datagramas quase a todo 
    momento.

\end{document}
