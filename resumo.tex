\documentclass[a4paper, 12pt]{article}

\usepackage[portuges]{babel}
\usepackage[utf8]{inputenc}
\usepackage{amsmath}
\usepackage{indentfirst}
\usepackage{graphicx}
\usepackage{multicol,lipsum}

\begin{document}
%\maketitle

\begin{titlepage}
	\begin{center}
	
	%\begin{figure}[!ht]
	%\centering
	%\includegraphics[width=2cm]{c:/ufba.jpg}
	%\end{figure}

		\large{Universidade Federal de Mato Grosso do Sul}\\
		\large{Campus de Ponta Porã}\\ 
		\vspace{15pt}
        \vspace{95pt}
        \textbf{\LARGE{Relatório - Camada de Rede }}\\
		%\title{{\large{Título}}}
		\vspace{3,5cm}
	\end{center}
	
	\begin{flushleft}
		\begin{tabbing}
			Aluno: Leandro Viana Martins\\
			Professor: Dr. Dionisio Machado Leite\\
	\end{tabbing}
 \end{flushleft}
	\vspace{1cm}
	
	\begin{center}
		\vspace{\fill}
			Setembro\\
            2019
	\end{center}
\end{titlepage}
%%%%%%%%%%%%%%%%%%%%%%%%%%%%%%%%%%%%%%%%%%%%%%%%%%%%%%%%%%%

\newpage
\section{Introdução}
Na camada de rede, são implementados serviços de comunicação hospedeiro a hospedeiro, diferentemente da camada de transporte. 
Portanto, há "pedaços" da camada em cada um de seus hospedeiros e roteadores na rede.
\\
\\
É necessário distinguir as funções de \textbf{repasse e roteamento}, onde o primeiro envolve a transferência de um pacote de um enlace
de entrada para um elance de saída. Já o segundo, envolve todos os roteadores de uma rede, onde os caminhos que os pacotes 
percorrem serão determinados por protocolos.

\subsection{Repasse e Roteamento}
O papel da camada de rede é \textbf{ transportar os pacotes de um hospedeiro origem até um hospedeiro destinatário}.

    \subsubsection{Repasse}
        Quando um pacote chega num roteador, este deve encaminhar até a saída apropriada.
        Dado um pacote — proveniente de um hospedeiro H1 — que chega ao roteador R1 com destino a um hospedeiro H2, deve ser 
        repassado ao roteador seguinte até chegar em seu destino.

    \subsubsection{Roteamento}
        A camada de rede deve determinar a rota em que um pacote flui de seu remetente à seu destinatário. Os responsáveis por essas
        atividads são os \textbf{algoritmos de roteamento}. Nesse caso, um algoritmo definiria o caminho entre dois hospedeiros em que
        os pacotes seguirão.

\subsection{Modelos de Serviço de rede}

Um modelo de serviço de rede define características pertinentes ao transporte dos dados \textbf{ fim a fim entre bordas da rede}.
Visto isso, o modelo poderia prover alguns serviços como: \textit{Entrega garantida, entrega garantida com atraso limitado, 
entrega de pacotes em ordem, largura de banda mínima garantida, Jitter e serviços de segurança.}

    \subsubsection{Arquitetura de rede: Internet}
    Possui o modelo de serviço \textit{melhor esforço}, com \textbf{nenhuma} garantia de \textit{largura de banda} e \textit{contra perda}.
    Seu \textit{ordenamento} é feito em \textbf{qualquer ordem possível} sem manter \textit{temporização} com \textbf{nenhuma} 
    \textit{indicação de congestionamento}

    \subsubsection{Arquitetura de rede: ATM}
    Possui o modelo de serviço \textit{CBR}, com \textbf{taxa constante} de garantia de \textit{largura de banda} e possuí 
    \textit{garantia contra perda}.
    Seu \textit{ordenamento} é feito em \textbf{ordem} mantendo \textit{temporização} com \textbf{garantia de que não ocorrerá} 
    \textit{congestionamento}

    \subsubsection{Arquitetura de rede: ATM}
    Possui o modelo de serviço \textit{ABR}, com \textbf{taxa mínima} de garantia de \textit{largura de banda} e \textbf{não possuí} 
    \textit{garantia contra perda}.
    Seu \textit{ordenamento} é feito em \textbf{ordem}, mas sem \textit{temporização} com \textbf{garantia de congestionamento} 

\newpage
\section{Redes de Circuitos Virtuais e Datagramas}
A camada de rede também oferece dois serviços para as aplicações, semelhantes aos que a camada de transporte oferece. Que são: serviços 
de camada de rede orientados para conexão e não orientados para conexão.
\\
\\
Um serviço orientado para conexão começa com uma apresentação entre os hospedeiros de origem e destino. Já o não orientado, não possui 
apresentação alguma.
\subsection{Redes de Circuitos Virtuais}
Um circuito virtual é um serviço orientado a conexão, onde trata-se basicamente de um caminho (série de enlaces e roteadores) entre 
hospedeiros de origem e destino. O qual contém um número para cada enlace e registros da tabela de repasse em cada roteador ao longo do caminho.

    \subsubsection{Estabelecimento do Circuito virtual}
    O estabelecimento de um circuito virtual é consistido por três fases:
        \\
        \\*
        I. \textbf{Estabelecimento de CV}: A camada de transporte remetente contata a camada de rede especificando o endereço do destinatário.
        Depois, a camada de rede determina o caminho entre o remetente e destinatário.
        \\
        \\*
        II. \textbf{Transferência de dados}: Fluxo dos pacotes entre os hospedeiros.
        \\
        \\*
        III. \textbf{Encerramento do CV}: O encerramento começa quando um dos hospedeiros avisa à camada de rede seu desejo de desativar o CV.
        A camada de rede então, informa ao sistema final do outro lado sob o fim da conexão e atualiza as tabelas de repasse de cada um dos
        roteadores.

\subsection{Rede de Datagramas}
Uma rede de Datagramas é um serviço não orientado a conexão, quando um sistema final quer enviar um pacote, ele informa no pacote
o endereço do de destino e posteriormente o envia pra dentro da rede. Nesse, cenário, a rota em que fluxo dos pacotes seguirão está definida
na tabela de repasse do roteador.

\end{document}
