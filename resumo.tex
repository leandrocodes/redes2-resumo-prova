\documentclass[a4paper, 12pt]{article}

\usepackage[portuges]{babel}
\usepackage[utf8]{inputenc}
\usepackage{amsmath}
\usepackage{indentfirst}
\usepackage{graphicx}
\usepackage{multicol,lipsum}

\begin{document}
%\maketitle

\begin{titlepage}
	\begin{center}
	
	%\begin{figure}[!ht]
	%\centering
	%\includegraphics[width=2cm]{c:/ufba.jpg}
	%\end{figure}

		\large{Universidade Federal de Mato Grosso do Sul}\\
		\large{Campus de Ponta Porã}\\ 
		\vspace{15pt}
        \vspace{95pt}
        \textbf{\LARGE{Relatório - Camada de Rede }}\\
		%\title{{\large{Título}}}
		\vspace{3,5cm}
	\end{center}
	
	\begin{flushleft}
		\begin{tabbing}
			Aluno: Leandro Viana Martins\\
			Professor: Dr. Dionisio Machado Leite\\
	\end{tabbing}
 \end{flushleft}
	\vspace{1cm}
	
	\begin{center}
		\vspace{\fill}
			Setembro\\
            2019
	\end{center}
\end{titlepage}
%%%%%%%%%%%%%%%%%%%%%%%%%%%%%%%%%%%%%%%%%%%%%%%%%%%%%%%%%%%

\newpage
\section{Introdução}
Na camada de rede, diferentemente da camada de transporte, são implementados serviços de comunicação hospedeiro a
hospedeiro. Portanto, há "pedaços" da camada em cada um de seus hospedeiros e roteadores na rede.



\end{document}
